\documentclass{article}
\usepackage[utf8]{inputenc}
\usepackage{enumitem}
\setlength{\parskip}{1em}
\setlength{\parindent}{0em}

\title{Introducción a CSS}
\author{}
\date{}

\begin{document}

\maketitle

\section{¿Qué es CSS y por qué debes aprenderlo?}
CSS (Cascading Style Sheets) es un lenguaje de estilo utilizado para describir la apariencia y el formato de un documento escrito en HTML. Al utilizar CSS, puedes controlar la apariencia de varios elementos de HTML en un solo archivo de estilo, en lugar de tener que escribir el estilo para cada elemento individualmente. Esto hace que sea más fácil y rápido modificar la apariencia de un sitio web, y también hace que sea más fácil mantener y actualizar el sitio a lo largo del tiempo.

Es importante aprender CSS porque es uno de los pilares de la creación de sitios web. Aunque puedes crear un sitio web básico usando solo HTML, CSS te permite darle estilo y diseño a tu sitio, lo que lo hace más atractivo y fácil de usar para los visitantes. Además, muchos trabajos relacionados con la creación de sitios web requieren conocimientos de CSS, por lo que aprender CSS puede ser beneficioso para tu carrera profesional.

\section{¿En dónde se utiliza CSS?}
CSS es utilizado en una amplia variedad de tecnologías y plataformas, incluyendo:
\begin{itemize}
	\item Sitios web: La mayoría de los sitios web modernos utilizan CSS para dar estilo y diseño a sus páginas.
	\item Aplicaciones móviles: Muchas aplicaciones móviles utilizan CSS para controlar la apariencia de sus interfaces de usuario.
	\item Aplicaciones de escritorio: Algunas aplicaciones de escritorio también pueden utilizar CSS para dar estilo a sus interfaces de usuario.
	\item Presentaciones: Algunas herramientas de presentación, como Microsoft PowerPoint, permiten utilizar CSS para dar estilo a las diapositivas.
	\item Correo electrónico: Algunos programas de correo electrónico, como Microsoft Outlook, permiten utilizar CSS para dar estilo a los mensajes de correo electrónico.
	\item Aplicaciones de publicación digital: Las aplicaciones de publicación digital, como Adobe InDesign, pueden utilizar CSS para dar estilo a los documentos que se crean con ellas.
\end{itemize}

\section{¿Qué trabajos puedes conseguir al aprender CSS?}
Al aprender CSS, podrías considerar trabajar como:
\begin{itemize}
	\item Diseñador/a web: Un diseñador/a web utiliza herramientas como HTML y CSS para crear y dar formato a sitios web.
	\item Desarrollador/a front-end: Un desarrollador/a front-end se encarga de la parte de un sitio web que los visitantes ven y interactúan, y utiliza tecnologías como HTML, CSS y JavaScript para crear la interfaz de usuario.
	\item Diseñador/a gráfico: Un diseñador/a gráfico puede utilizar CSS, junto con otras herramientas de diseño, para crear elementos visuales para sitios web y otras plataformas.
	\item Diseñador/a de correo electrónico: Un diseñador/a de correo electrónico se encarga de crear y diseñar mensajes de correo electrónico atractivos y bien diseñados que sean fáciles de leer y navegar. Pueden utilizar CSS para dar estilo a estos mensajes.
	\item Diseñador/a de aplicaciones móviles: Un diseñador/a de aplicaciones móviles utiliza herramientas como CSS para crear y dar formato a las interfaces de usuario de las aplicaciones móviles.
\end{itemize}

\section{¿Cuánto puedes ganar usando CSS en tu trabajo?}
El salario que puedes ganar al utilizar CSS en tu trabajo dependerá de varios factores, como tu nivel de experiencia, tu ubicación geográfica y el tipo de trabajo que realices. En general, el salario de las personas que utilizan CSS en su trabajo puede variar ampliamente, desde unos pocos miles de dólares al año hasta varios cientos de miles de dólares al año.

\subsection{Ejemplos de Salarios Promedio}
\begin{itemize}
	\item Diseñador/a web: El salario promedio para un diseñador/a web en Estados Unidos es de alrededor de \$50,000 al año.
	\item Desarrollador/a front-end: El salario promedio para un desarrollador/a front-end en Estados Unidos es de alrededor de \$75,000 al año.
	\item Diseñador/a gráfico: El salario promedio para un diseñador/a gráfico en Estados Unidos es de alrededor de \$50,000 al año.
	\item Diseñador/a de correo electrónico: El salario promedio para un diseñador/a de correo electrónico en Estados Unidos es de alrededor de \$50,000 al año.
	\item Diseñador/a de aplicaciones móviles: El salario promedio para un diseñador/a de aplicaciones móviles en Estados Unidos es de alrededor de \$80,000 al año.
\end{itemize}

\section{Preguntas Comunes sobre CSS}
Aquí hay algunas preguntas comunes sobre CSS:
\begin{enumerate}
	\item ¿Qué es CSS y para qué se utiliza?
	\item ¿Cuál es la diferencia entre CSS y HTML?
	\item ¿Cuáles son algunas de las principales propiedades de CSS?
	\item ¿Cómo puedo utilizar CSS para dar estilo a los elementos de un sitio web?
	\item ¿Cómo puedo hacer que mi sitio web sea responsive con CSS?
	\item ¿Cómo puedo utilizar la hoja de estilo en cascada (CSS) para crear animaciones?
	\item ¿Cómo puedo utilizar media queries en CSS para crear diseños móviles?
	\item ¿Cómo puedo utilizar CSS para crear efectos de hover en los enlaces?
	\item ¿Cómo puedo utilizar CSS para crear diseños de grillas y maquetas?
	\item ¿Cómo puedo utilizar CSS para crear diseños adaptativos y cambiar la apariencia de un sitio web según el tamaño de la pantalla?
\end{enumerate}

\end{document}
