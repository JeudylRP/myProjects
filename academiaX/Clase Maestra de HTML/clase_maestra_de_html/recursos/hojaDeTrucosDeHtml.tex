


\documentclass{article}
\usepackage[utf8]{inputenc}
\usepackage{array}
\usepackage{longtable}
\usepackage{geometry}

\geometry{
    a4paper,
    left=20mm,
    right=20mm,
    top=25mm,
    bottom=25mm,
}

\title{Hoja de trucos de HTML}
\author{}
\date{}

\begin{document}

\maketitle

\section*{1. Estructura básica de un documento HTML}
\begin{tabular}{>{\ttfamily}l p{10cm}}
\textnormal{Elemento} & Descripción \\
\hline
<!DOCTYPE html> & Declara el tipo de documento y la versión de HTML. \\
<html> & Elemento raíz de un documento HTML. \\
<head> & Contiene metadatos y enlaces a scripts y hojas de estilo. \\
<title> & Especifica el título de la página. \\
<body> & Contiene el contenido visible de la página web. \\
\end{tabular}

\section*{2. Elementos de encabezado}
\begin{tabular}{>{\ttfamily}l p{10cm}}
\textnormal{Elemento} & Descripción \\
\hline
<h1> & Encabezado más grande. \\
<h2> & Subencabezado. \\
<h3> a <h6> & Encabezados de menor importancia. \\
\end{tabular}

\section*{3. Párrafos y formateo de texto}
\begin{tabular}{>{\ttfamily}l p{10cm}}
\textnormal{Elemento} & Descripción \\
\hline
<p> & Define un párrafo. \\
<strong> & Indica texto en negrita. \\
<em> & Indica texto en itálica. \\
<br> & Inserta un salto de línea. \\
<hr> & Crea una línea horizontal de separación. \\
\end{tabular}

\section*{4. Listas}
\begin{tabular}{>{\ttfamily}l p{10cm}}
\textnormal{Elemento} & Descripción \\
\hline
<ul> & Define una lista no ordenada. \\
<li> & Define un ítem de la lista. \\
<ol> & Define una lista ordenada. \\
\end{tabular}

\section*{5. Enlaces e imágenes}
\begin{tabular}{>{\ttfamily}l p{10cm}}
\textnormal{Elemento} & Descripción \\
\hline
<a href="URL"> & Crea un enlace a otra página. \\
<img src="imagen.jpg" alt="Descripción"> & Incorpora una imagen. \\
\end{tabular}

\section*{6. Tablas}
\begin{tabular}{>{\ttfamily}l p{10cm}}
\textnormal{Elemento} & Descripción \\
\hline
<table> & Define una tabla. \\
<tr> & Define una fila de la tabla. \\
<th> & Define una celda de encabezado. \\
<td> & Define una celda de tabla. \\
\end{tabular}

\section*{7. Formularios}
\begin{tabular}{>{\ttfamily}l p{10cm}}
\textnormal{Elemento} & Descripción \\
\hline
<form> & Define un formulario para la entrada del usuario. \\
<input> & Define un campo de entrada. \\
type="text" & Campo de texto. \\
type="password" & Campo de contraseña. \\
type="submit" & Botón para enviar el formulario. \\
\end{tabular}

\section*{8. Elementos semánticos}
\begin{tabular}{>{\ttfamily}l p{10cm}}
\textnormal{Elemento} & Descripción \\
\hline
<header> & Representa un grupo de soporte introductorio o navegacional. \\
<nav> & Sección de enlaces de navegación. \\
<article> & Contenido independiente. \\
<section> & Sección de un documento. \\
<footer> & Pie de página de un documento o sección. \\
\end{tabular}

\section*{9. Atributos globales}
\begin{tabular}{>{\ttfamily}l p{10cm}}
\textnormal{Atributo} & Descripción \\
\hline
id="identificadorUnico" & Identificador único para el elemento. \\
class="listaDeClases" & Lista de clases para estilizar con CSS. \\
style="propiedad:valor;" & Estilos CSS directamente en el elemento. \\
data-*="dato" & Almacena datos personalizados. \\
title="informaciónAdicional" & Información que se muestra como tooltip. \\
\end{tabular}

\end{document}




